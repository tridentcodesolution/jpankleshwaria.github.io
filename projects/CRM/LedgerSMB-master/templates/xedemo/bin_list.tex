<?lsmb FILTER latex -?>
\documentclass{scrartcl}
 \usepackage{xltxtra}
 \usepackage{fontspec}
 \setmainfont{LiberationSans-Regular.ttf}
\usepackage{tabularx}
\usepackage{longtable}
\usepackage[letterpaper,top=2cm,bottom=1.5cm,left=1.1cm,right=1.5cm]{geometry}
\usepackage{graphicx}

\begin{document}

\pagestyle{myheadings}
\thispagestyle{empty}

<?lsmb LETTERHEAD ?>


% Breaking old pagebreak directives
%<?xlsmb pagebreak 65 27 37 ?>
%\end{tabularx}
%
%\newpage
%
%\markboth{<?xlsmb company ?>\hfill <?xlsmb ordnumber ?>}{<?xlsmb company ?>\hfill <?xlsmb ordnumber ?>}
%
%\begin{tabularx}{\textwidth}{@{}rlXllrrll@{}}
%  \textbf{Item} & \textbf{Number} & \textbf{Description} & \textbf{Serial Number} & & \textbf{Qty} & \textbf{Recd} & & \textbf{Bin} \\
%<?xlsmb end pagebreak ?>


\vspace*{0.5cm}

\parbox[t]{.5\textwidth}{
\textbf{<?lsmb text('From') ?>}
\vspace{0.3cm}

<?lsmb name ?>

<?lsmb address1 ?>

<?lsmb address2 ?>

<?lsmb city ?>
<?lsmb IF state ?>
\hspace{-0.1cm}, <?lsmb state ?>
<?lsmb END ?>
<?lsmb zipcode ?>

<?lsmb country ?>
}
\parbox[t]{.5\textwidth}{
\textbf{<?lsmb text('Ship To') ?>}
\vspace{0.3cm}

<?lsmb shiptoname ?>

<?lsmb shiptoaddress1 ?>

<?lsmb shiptoaddress2 ?>

<?lsmb shiptocity ?>
<?lsmb IF shiptostate ?>
\hspace{-0.1cm}, <?lsmb shiptostate ?>
<?lsmb END ?>
<?lsmb shiptozipcode ?>

<?lsmb shiptocountry ?>
}
\hfill

\vspace{1cm}

\textbf{\MakeUppercase{<?lsmb text('Bin List') ?>}}
\hfill

\vspace{1cm}

\begin{tabularx}{\textwidth}{*{6}{|X}|} \hline
  \textbf{Order \#} & \textbf{Date} & \textbf{Contact}
  <?lsmb IF warehouse ?>
  & \textbf{<?lsmb text('Warehouse') ?>}
  <?lsmb END ?>
  & \textbf{<?lsmb text('Shipping Point') ?>} & \textbf{<?lsmb text('Ship via') ?>} \\ [0.5em]
  \hline

  <?lsmb ordnumber ?>
  <?lsmb IF shippingdate ?>
  & <?lsmb shippingdate ?>
  <?lsmb END ?>
  <?lsmb IF NOT shippingdate ?>
  & <?lsmb orddate ?>
  <?lsmb END ?>
  & <?lsmb employee ?>
  <?lsmb IF warehouse ?>
  & <?lsmb warehouse ?>
  <?lsmb END ?>
  & <?lsmb shippingpoint ?> & <?lsmb shipvia ?> \\
  \hline
\end{tabularx}

\vspace{1cm}

\begin{longtable}{@{\extracolsep{\fill}}rllllrrll@{}}
  \textbf{<?lsmb text('Item') ?>} & \textbf{<?lsmb text('Number') ?>}
     & \textbf{<?lsmb text('Description') ?>} &
     \textbf{<?lsmb text('Serial Number') ?>} &
    & \textbf{<?lsmb text('Qty') ?>} & \textbf{<?lsmb text('Recd') ?>} &
    & \textbf{<?lsmb text('Bin') ?>} \\

<?lsmb FOREACH number ?>
<?lsmb lc = loop.count - 1 ?>
  <?lsmb runningnumber.${lc} ?> &
  <?lsmb number.${lc} ?> &
  <?lsmb item_description.${lc} ?> &
  <?lsmb serialnumber.${lc} ?> &
  <?lsmb deliverydate.${lc} ?> &
  <?lsmb qty.${lc} ?> &
  <?lsmb ship.${lc} ?> &
  <?lsmb unit.${lc} ?> &
  <?lsmb bin.${lc} ?> \\
<?lsmb END ?>
\end{longtable}


\rule{\textwidth}{2pt}

\end{document}
<?lsmb END ?>
